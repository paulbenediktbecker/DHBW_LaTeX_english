%%% Präambel %%%
% hier sollten keine Änderungen erforderlich sein
%
\usepackage[utf8]{inputenc}   % Zeichencodierung UTF-8 für Eingabe-Dateien
\usepackage[T1]{fontenc}      % Darstellung von Umlauten im PDF

\usepackage{listings}         % für Einbindung von Code-Listings
\lstset{numbers=left,numberstyle=\tiny,numbersep=5pt,texcl=true}
\lstset{literate=             % erlaubt Sonderzeichen in Code-Listings 
{Ö}{{\"O}}1
{Ä}{{\"A}}1
{Ü}{{\"U}}1
{ß}{{\ss}}2
{ü}{{\"u}}1
{ä}{{\"a}}1
{ö}{{\"o}}1
{€}{{\euro}}1
}

\usepackage[
  inner=35mm,outer=15mm,top=25mm,
  bottom=20mm,foot=12mm,includefoot
]{geometry}                 % Einstellungen für Ränder

%\usepackage[ngerman]{babel} % Spracheinstellungen Deutsch
%\usepackage[babel,german=quotes]{csquotes} % deutsche Anf.zeichen
\usepackage[english]{babel} % Spracheinstellungen Englisch
\usepackage[babel,english=british]{csquotes} % englische Anf.zeichen
\usepackage{enumerate}      % anpassbare Nummerier./Aufz.
\usepackage{enumitem}
\usepackage{graphicx}       % Einbinden von Grafiken
\usepackage[onehalfspacing]{setspace} % anderthalbzeilig

\usepackage{blindtext}      % Textgenerierung für Testzwecke
\usepackage{color}          % Verwendung von Farbe 
\usepackage{float}
\usepackage{acronym}        % für ein Abkürzungsverzeichnis
\usepackage{comment}
\usepackage[                % Biblatex
  backend=biber,
  bibstyle=_dhbw_authoryear,maxbibnames=99,
  citestyle=authoryear,     
  uniquename=true, useprefix=true,
  bibencoding=utf8]{biblatex}
\input{template/_dhbw_biblatex-config.tex} % mit DHBW-spezifischen Einstellungen

% If you want to break on URL numbers
\setcounter{biburlnumpenalty}{9000}
% If you want to break on URL lower case letters
\setcounter{biburllcpenalty}{9000}
% If you want to break on URL UPPER CASE letters
\setcounter{biburlucpenalty}{9000}

%\usepackage{xurl}
%\PassOptionsToPackage{hyphens}{url}
\usepackage[hidelinks]{hyperref}
\def\UrlBreaks{\do\/\do-}



\def\hyphenateAndTtWholeString #1{\xHyphenate#1$\wholeString\unskip}

\def\xHyphenate#1#2\wholeString {\if#1$%
	\else\transform{#1}%
	\takeTheRest#2\ofTheString\fi}

\def\takeTheRest#1\ofTheString\fi
{\fi \xHyphenate#1\wholeString}

\def\transform#1{\url{#1}\hskip 0pt plus 1pt}

% Define the \urlx command which works like \url, but with line brakes
\def\urlx #1{\href{#1}{\hyphenateAndTtWholeString{#1}}}

\usepackage{tocloft}        % für Verzeichnis der Anhänge

\usepackage{pdfpages}
\usepackage{gensymb}		% für \degree

\newcounter{anhcnt}
\setcounter{anhcnt}{0}
\newlistof{anhang}{app}{}

\newcommand{\anhang}[1]{%
  \refstepcounter{anhcnt}
  \setcounter{anhteilcnt}{0}
  \section*{Appendix \theanhcnt: #1}
  \addcontentsline{app}{section}{\protect\numberline{Appendix \theanhcnt}#1}\par
}

\newcounter{anhteilcnt}
\setcounter{anhteilcnt}{0}

\newcommand{\anhangteil}[1]{%
	\refstepcounter{anhteilcnt}
	\subsection*{Appendix~\arabic{anhcnt}/\arabic{anhteilcnt}: #1}
	\addcontentsline{app}{subsection}{\protect\numberline{Appendix \theanhcnt/\arabic{anhteilcnt}}#1}\par
}

\renewcommand{\theanhteilcnt}{Appendix \theanhcnt/\arabic{anhteilcnt}}

% vgl. S. 4 Paket-Beschreibung tocloft 	
% Einrückungen für Anhangverzeichnis
\makeatletter
\newcommand{\abstaendeanhangverzeichnis}{
\renewcommand*{\l@section}{\@dottedtocline{1}{0em}{5.5em}}
\renewcommand*{\l@subsection}{\@dottedtocline{2}{2.3em}{6.5em}}
}
\makeatother

% Abbildungs- und Tabellenverzeichnis
% Bezeichnungen
%\renewcaptionname{ngerman}{\figurename}{Abb.}
%\renewcaptionname{ngerman}{\tablename}{Tab.}
%\addto\captionsenglish{english}{\figurename}{Fig.}
%\addto\captionsenglish{english}{\tablename}{Tab.}

% Einrückungen
\makeatletter
\renewcommand*{\l@figure}{\@dottedtocline{1}{0em}{2.3em}}
\renewcommand*{\l@table}{\@dottedtocline{1}{0em}{2.3em}}
\makeatother


\usepackage{chngcntr}                % fortlaufende Zähler für Fußnoten, Abbildungen und Tabellen
\counterwithout{figure}{chapter}
\counterwithout{table}{chapter}
\counterwithout{footnote}{chapter}

\usepackage[automark]{scrlayer-scrpage} 
\input{template/_dhbw_kopfzeilen.tex}		 % für schöne Kopfzeilen 

\usepackage{textcomp}            % erlaubt EUR-Zeichen in Eingabedatei
\usepackage{eurosym}             % offizielles EUR-Symbol in Ausgabe
\renewcommand{\texteuro}{\euro}  % ACHTUNG: nach hyperref aufrufen!

\usepackage{scrhack}             % stellt Kompatibilität zw. KOMA-Script
                                 % (scrreprt) und anderen Paketen her
                                 
% Anpassung der Abstände bei Kapitelüberschriften
% (betrifft auch Inhalts-, Abbildungs- und Tabellenverzeichnis)
\renewcommand*\chapterheadstartvskip{\vspace*{-\topskip}}
\newcommand{\myBeforeTitleSkip}{1mm}
\newcommand{\myAfterTitleSkip}{10mm}
\setlength\cftbeforetoctitleskip{\myBeforeTitleSkip}
\setlength\cftbeforeloftitleskip{\myBeforeTitleSkip}
\setlength\cftbeforelottitleskip{\myBeforeTitleSkip}

\setlength\cftaftertoctitleskip{\myAfterTitleSkip}
\setlength\cftafterloftitleskip{\myAfterTitleSkip}
\setlength\cftafterlottitleskip{\myAfterTitleSkip}  

%\DefineBibliographyStrings{english}{andothers={{}}}   
                                                    
%%% Ende der Präambel %%%